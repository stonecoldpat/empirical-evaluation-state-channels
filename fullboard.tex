\section{Battleship Contract}
	
	
We provide a high-level overview of the game battleship before proposing how to implement it as a smart contract.  
A security analysis for the game is included in Appendix \ref{sec:secanalysis}.
We present how to convert the battleship game to support state channels using the template in Section \ref{sec:template}. 
	
\subsection{Overview of Battleship}

Battleship is a two-player game where each player has a list of ships that are placed on a 10x10 private board. 
Each ship must be marked in a straight line either horizontally or vertically.
Our protocol only relies on a commitment to every player's ship and the signed messages exchanged between both parties in order to minimise long-term storage (and the associated gas-cost).
An extension to this game is presented in Appendix \ref{sec:commitmentcell} which includes a commitment for every cell on the board. 

%However it increases the gas cost for playing the game the only benefit is to p discuss how to extend this game to include a commitment for every cell in Appendix \ref{sec:commitmentcell}, but it increases the gas cost to deploy and play the game. 

To set up the game, both parties exchange a commitment to their list of ships and the counterparty must submit it using $\battleshipselectboard$. 
Afterwards both players can signal to begin the game using $\battleshipbegin$, otherwise they can quit using $\battleshipgameover$. 
In the turn-based gameplay, the player selects a cell to shoot using $\battleshipattackcell$ and the counterparty must open the cell  within a fixed challenge period. 
To open, the counterparty reveals if the cell is occupied by water or a ship piece using $\battleshiprevealcell$.
If this shot sinks a full ship, then the counterparty must reveal the full ship (i.e. instead of the cell's opening) using $\battleshipsinking$. 
The player ican take another turn if their shot was successful.
At the end, the winner must reveal their board and every ship's location to the loser using $\battleshiprevealboard$.
The loser has a fixed challenge period to prove if the winner's board was incorrectly set up or if the winner cheated during the game using a proof of fraud. 
A player can call $\battleshipgameover$ after the challenge period has expired to finish the game. 

\subsection{Battleship Contract}\label{sec:battleshipcontract}

We present each phase of the game, how to establish the contract, the turn-based gameplay and finally how the loser is provided an opportunity to prove the winner cheated.

\paragraph{Game Phases} There are six phases $\gamesetup, \gameattack, \gamereveal,  \gamewinner, \gamefraud, \\ \gamefinished$. 
The $\gamesetup$ phase is responsible for ensuring both players select a single list of ships to begin the game. 
Game play transitions between $\gameattack$ and $\gamereveal$ as both players take a turn at shooting the counterparty's ships. 
The game transitions to $\gamewinner$ when one player wins the game and it will transition to $\gamefraud$ once the winner has opened all ship locations.
This provides the loser a fixed time period to submit a proof of fraud that the winner's board is not well-formed or that the winner did not honestly reveal a cell during the game. 
Otherwise, the contract transitions to $\gamefinished$ and the winner can claim their winnings.
% to open  to declare a cell, otherwise the contract transitions to $\gamefinished$. 
%Of course, the contract transitions to $\gamefraud$ if a proof of fraud to detect cheating by the counterparty is provided and we explore this phase further in Section \ref{sec:fraud}. 

\paragraph{Contract establishment} 
The contract is established with the address of both players $\participant_{1},\participant_{2}$ and the challenge timer $\timerchallenge$. 
Both parties can deposit coins during $\gamesetup$ phase before placing their bets.

\paragraph{Prepare list of ships} %Both parties participate in a cut-and-choose protocol during $\gamestatus := \gamesetup$. 
A ship hash is denoted as $\hship := \hash(x,y,x',y',r, \mathsf{round}, \participant, \appcontract)$ where $x,y$ represents its starting co-ordinate, $x',y'$ represents its finishing co-ordinate. 
Each party $\participant$ computes and signs a list of ships: 

\begin{center}
	$\Sigma_{1}^{N} := \sign_{\participant}(((k_{1},\hship_{1}),...,(k_{n},\hship_{n})), \participant, \mathsf{round}, \appcontract)$ 
\end{center}

Each ship in the list is denoted as $(k,\hship)$, where $k$ is the length for that particular ship.  
This is sent to the counterparty who must submit it using $\battleshipselectboard$ and reserve the ships for the game.\footnote{In Appendix  \ref{sec:cutandchoose} we present a cut-and-choose protocol to allow the counterparty probabilistic verify the board is well-formed.}
Both players can notify the contract to begin the game using $\battleshipbegin$ or one party can signal their desire to quit using $\battleshipgameover$.
Finally the game $\mathsf{round}$ is incremented regardless if it continues or not. 

\paragraph{Game-play} \label{sec:gameplayships}
%The game phase transitions several times between $\gameattack$ and $\gamereveal$ as each player has a turn at shooting the counterparty's ships. 
The contract maintains a counter $\mathsf{move}$ which is incremented after each player's turn. 
In the $\gameattack$ phase, the player $\participant$ challenges the counterparty to open a cell $x,y$ by signing: 

\begin{center}
	$\sigma^{shot}_{\participant} := \sign_{\participant}(x,y, \mathsf{move}, \mathsf{round},\appcontract)$ \\
\end{center}

This message is submitted using $\battleshipattackcell$.
It transitions the game phase to $\gamereveal$ and sets a fixed challenge period $\timechallenge := \timenow + \timerchallenge$ for the counterparty's response. 
The counterparty signs one of two messages depending on whether a ship was sunk:

\begin{center}
	$\sigma^{hit}_{\participant} := \sign_{\participant}(x,y,b,\mathsf{move}, \mathsf{round},\appcontract)$ \\ $\sigma^{sunk}_{\participant} := \sign_{\participant}(x,y,x',y',r,\hship,\mathsf{move}, \mathsf{round},\appcontract)$
\end{center}

The counterparty is responsible for submitting either signed message. 
The first message declares if the cell is marked with water $(b=0)$ or a ship location $(b=1)$.
It is submitted using $\battleshiprevealcell$.
The second message declares the shot sank a ship and requires the counterparty to open the corresponding ship commitment $\hship$ to $\battleshipsinking$. 
Each party must keep a copy of every signed message\footnote{Every signed message is emitted by the contract and thus it is easily fetchable.} as it can later be used to prove fraud which we discuss in Section \ref{sec:prooffraud}.
The game transitions to $\gamewinner$ if one player has declared all their ships sunk. 

\paragraph{End of game} 
After one player has lost the game (or if the contract has detected cheating by the loser as illustrated in Section \ref{sec:fraud}), the winner must open their remaining ship commitments using $\battleshiprevealships$.
This contract transitions to $\gamefraud$ which provides a fixed challenge period for the loser to submit a proof of fraud. 
After this time period, the winner can redeem their reward using $\battleshipgameover$  and the game  transitions to $\gamefinished$. 
Of course, if both parties have cheated, then the winnings are simply burnt. 

\subsection{Checking for Fraud} \label{sec:fraud}

We present integrity checks the contract can perform throughout the game to verify that either party has not cheated. 
These checks are performed whenever a player calls $\battleshipattackcell, \battleshipsinking, \battleshiprevealcell$ and $\battleshiprevealships$.


\paragraph{Exceeded maximum number of moves} 
The contract maintains three counters.
The first $\mathsf{move}$ keeps track of the number of actions taken by bother players.
If $\mathsf{move}$ exceeds the number of possible moves in the game for both players, then the contract can confirm that both players have cheated as an honest player will have declared all their ships as sunk before the limit for $\mathsf{move}$ is exceeded.
In this case, both players are set as cheating and the game transitions to $\gamefinished$ without a winner. 
Both $\mathsf{hits}_{i}$ and $\mathsf{water}_{i}$ keeps track of each player's attack on the counterparty's board. 
If $\mathsf{hits}_{i}$ exceeds the number of ship positions on the board or $\mathsf{water_{i}}$ exceeds the possible number of water cells, then the counterparty was dishonest about their cell opening. 
In this case, the counterparty is marked as cheated, the game transitions to $\gamewinner$ and the winner must open their ships.  

\paragraph{Players only play using valid cells}  
All cells must be within the permitted range $0 <= x < 10$ and $0 <= y < 10$ for any signed message received. 

\paragraph{A ship was not placed  horizontally or vertically}\label{sec:vertical}
The contract can check whether an opened ship was placed on the board horizontally or vertically. 
To verify, it checks that every location for a ship either has the same $x$ or $y$ co-ordinate, and that $x$ or $y$ is incremented (or decremented) strictly by one for every ship location. 
It also checks the ship's length which is established during set up. 

\subsection{Proof of Fraud}  \label{sec:prooffraud}

To alleviate the need to validate the entire game within the smart contract environment (and incurring unreasonable gas costs), the protocol is designed to let each player validate the game and submit a proof of fraud if the counterparty has cheated. 
In the following we present the fraud proofs that can be verified by the contract. 

\paragraph{Player has shot the same cell twice} 
The contract cannot independently verify if a player has shot the same cell twice as it does not store the opening of cells.
Instead the counterparty can submit the two signed shots $\sigma^{shot}_{\participant},\sigma^{shot'}_{\participant}$, the corresponding $\mathsf{move},\mathsf{move}'$ counters and the cell $x,y$ using $\battleshipsamecell$. 
The contract verifies if the signatures are valid (and from the same party), both shots are for the same cell, and $\mathsf{move}\neq\mathsf{move}'$. 
This proof of fraud can be submitted to the contract at any point during the game. 

\paragraph{Counterparty was dishonest about a cell opening}
The counterparty has marked a cell $(x,y)$ as water, but an opened $\hship$ states it is a ship location. 
%As previously mentioned, the contract only stores ship openings and the player is required to store every signed cell opening. 
To prove fraud, the player submits the ship identifier $\hship$, the disputed cell $x,y$ and the signed opening of the cell $\sigma^{hit}_{\participant}$ using $\battleshipdeclarednothit$.
The contract can verify if this cell opening was signed by the counterparty as $b = 0$ and the ship $\hship$ claims to be at $x,y$.   
On the other hand, the counterparty may also mark a cell as a ship location,  but no ships are at that location.
This proof of fraud is similar as the player submits the disputed cell location $x,y$ alongside its signed opening $\sigma^{hit}_{\participant}$ using $\battleshipdeclarednotwater$. 
The contract is satisified if it cannot find a ship at that location. 
Both proofs can only be submitted during $\gamefraud$. 

%\paragraph{Adjacent ships} 
%This fraud proof proves that a ship is not surrounded by water and instead two ships are adjacent to each other (or claim to be at the same location on the board). 
%We only require the ship commitments $\hship_{1},\hship_{2}$ to be opened and the same proof can be used in both contracts. 
%Any party can submit the location $i,j$ for the first ship and $i',j'$ for the second ship.  
%The contract can verify if ships are adjacent by checking whether $i,j$ is within the range $i''1, j''1$ or if they claim to be at the same location $i,j = i',j'$.


\paragraph{Two ships claim to be at  the same cell} 
The cheater has used the same cell for two or more ships.  
The index for both ships and the cell $x,y$ must be submitted to the contract using $\battleshiptwoships$. 
The contract looks up the co-ordinates for each ship and checks if it claims to be at the same location $x, y$.
This proof is applicable during $\gamefraud$ after all ships are opened by the winner. 

\paragraph{Ship was not declared as sunk}
The counterparty did not declare a ship as sunk. 
All signed cell openings $\sigma^{hit}_{\participant,1},...,\sigma^{hit}_{\participant,k}$ and the ship identifier $\hship$ must be submitted to the contract using $\battleshipdeclarednotsunk$. 
This allows the contract to verify that every ship location was opened and this implies the counterparty did not declare the ship as sunk as the final opening should be $\sigma^{sunk}_{\participant}$. 
This proof is applicable during $\gamefraud$ after all ships are opened by the winner. 

\paragraph{Challenge period has expired }
The contract relies on a global clock (i.e. block timestamp or block height) for the challenge period  $\timerchallenge$. 
If a player does not respond within this time period, then the counterparty can notify the contract using $\battleshipchallengeexpired$ and the counterparty is set as the winner if the challenge period has expired. 


\section{Full Board Extension} \label{sec:commitmentcell}

We present an extension to our battleship game which requires a commitment for every cell of the board in addition to the ship commitments. 
For each cell, the commitment consists of a flag $b$ indicating if it is occupied by water $b := 0$ or a ship $b := 1$, and a randon nonce $r$. 
Assuming the selected board is well-formed, then it can prevent each player lying about their cell opening during the game, but it also increases the game state and the gas requirement for each move in the game. 

\subsection{Modifications to the Battleship Contract} 

We present how to modify the battleship contract to support the full board extension. 
This requires modifying how the game is prepared, how a cell opening during the game play is verified by the contract and how the full board is opened at the game's end. 

\paragraph{Prepare boards}
%Both parties still participate in a cut-and-choose protocol during $\gamestatus := \gamesetup$ and compute $n$ lists of ship hashes. 
Our extension requires each party to compute an entire board to accompany a list of ships.   
The board is a list of cell hashes such that $\hcell_{1,1},$ $...,\hcell_{n,n}$ where $n,n$ is the final grid co-ordinate. 
A cell hash is $\hash(b, r, \mathsf{round}, \participant, \appcontract)$, where $b$ is a flag indicating if it is occupied by water $b := 0$ or a ship location $b := 1$, and $r$ is the nonce. 
The party signs the list of ships and the board cells: 

\begin{displaymath}
 \sigma := \sign_{\participant}(((k_{1},\hship_{1}),...,(k_{n},\hship_{n})), (\hcell_{1},...,\hcell_{n}), \participant, \mathsf{round}, \appcontract)
\end{displaymath}
 
The contract stores every cell hash in the contract for future use. 
Each party is responsible for reserving the list of ships and the board on behalf of their counterparty using  $\battleshipselectboard$. 
All remaining $N-1$ list of ships and their corresponding boards must be opened and reviewed by the counterparty.  
If satisified, each party notifies the contract to begin the game using $\battleshipbegin$ or they can quit using $\battleshipgameover$.
%s part of the cut-and-choose protocol proposed for the original contract. 
%Finally the game $\mathsf{round}$ is incremented regardless if it continues or not. 

\paragraph{Game-play}
Our extension requires modifying how a player responds to an attacked cell:

\begin{center}
	$\sigma^{hit}_{\participant} := \sign_{\participant}(x,y,b, r_{cell},\mathsf{move}, \mathsf{round},\appcontract)$ \\ $\sigma^{sunk}_{\participant} := \sign_{\participant}(x,y,x',y',r_{cell}, r_{ship},\hship,\mathsf{move}, \mathsf{round},\appcontract)$
\end{center}

Both the hit and sunk messages include the nonce $r_{cell}$.
This lets the contract open $\hcell_{x,y}$ and confirm that the supplied $b$ matches the commitment during the setup.
The opening can be stored by the contract, otherwise each party must keep a copy of every signed message\footnote{Every signed message is emitted by the contract and thus it is easily fetchable.} for future fraud proofs as presented in Section \ref{sec:prooffraud}.


\subsection{Changes to Fraud Detection}

We present the additional fraud detection that is performed by the contract and the player due to the extension. 


\paragraph{Cut-and-choose protocol} \label{sec:cutandchoose}
To set up the game, both parties participate in a cut-and-choose protocol to provide a probabilistic guarantee that the counterparty's board is well-formed.
Each player commits, signs and sends the counterparty $N$ boards (i.e. list of ship and cell commitments).
The counterparty reserves one board for the game using $\battleshipselectboard$.
Once selected, each player reveals the remaining $N-1$ boards to the counterparty who verifies the boards are well-formed. 
If both parties are satisified, then they can signal to begin the game using $\battleshipbegin$, otherwise they can quit using $\battleshipgameover$. 
While this provides a probabilistic guarantee the board is correctly set up, it does not let each player place the ships on their board which may remove an element of the game. 

\paragraph{Integrity checks} 

As presented in Section~\ref{sec:fraud} the contract checks all signed messages received to self-enforce the game's correct execution. 
Our extension requires the contract to check every cell opening with the stored cell hash $\hcell$. 


%Our extension 
%In addition to the integrity checks presented in Section~\ref{sec:fraud} (\emph{Exceeded maximum number of moves}, \emph{Players only play using valid cells}, \emph{A ship was not placed horizontally or vertically}) that are performed in the same way, the cell commitments also allow checking for \emph{False Cell Reveal}, i.e. the contract ensures that a player cannot claim that a cell contains water if it in fact contains a ship by checking that the reveal matches the cell commitment (as described above). This allows ending a game early if a player tries to cheat by revealing false information about a cell. In our main version without cell commitments this could only be detected and punished at the end of the game.

\paragraph{Proof of fraud} 
If we assume the board is well-formed upon set up, then the party cannot be dishonest about their cell opening during the game. 
The fraud proofs $\battleshipdeclarednothit$ or $\battleshipdeclarednotwater$ are still required as the board used in the game can be invalid and the contract must verify that the cell opening does not correspond to a ship opening. 
There is no change to the fraud proof except that the cell nonces are submitted to the contract alongside the signed cell openings. 